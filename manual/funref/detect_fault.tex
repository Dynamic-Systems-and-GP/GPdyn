\subsection*{detect\_fault} \label{fun:detect_fault}


\textbf{Syntax}
\\  \texttt{function [index\_bool, index\_val] = detect\_fault(residuals, var\_residuals, NW,
 \\ \tab c\_max)}

\textbf{Description}
\\ Function for detecting the fault, where the system is modelled with the
 GP model. It takes into account predicted mean as well as the
predicted
 variance. Greater window size can be selected to increase
robustness on
 the account of lower sensibility.
\\ Note:
\\Function is not compatible with the current version of the GPML
 toolbox.
\\ {[}1{]} Dj.Juricic, J.Kocijan,Fault detection based on Gaussian process model,
\\ I.Troch, F.Breitenecker (eds.), Proceedings of the 5th Vienna
\\ Symposium on Mathematical Modelling � MathMod, Wien, 2006
\\
\\ Inputs:
\\ residuals .. [eps(1) .. eps(n)], where eps(k) = $y(k)-\hat y(k) =
y(k)-m(k)$
\\ var\_residuals .. [v(1) .. v(n)], where w(k) = var(eps(k))  predicted
     variance associated \\ \tab with $\hat y(k)$, i.e. GP prediction at time
step k
\\ NW .. size of window, used for calculating the index and deciding about fault
\\ c\_max .. maximal significance level still reflecting in deciding for H0
  (no fault)
\\ Outputs:
\\ index\_bool: 1 for H1 (fault) and 0 for H0 (no fault)
\\ index\_val: significance levels, corresponding to index\_bool


\textbf{See Also}
\\ DETECT\_FAULT\_VALIDITY, SIMUL00MCMC
