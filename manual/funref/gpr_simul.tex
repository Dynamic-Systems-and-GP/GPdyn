\subsection*{gpr\_simul} \label{fun:gpr_simul}


\textbf{Syntax}
\\ \texttt{function [out1, out2, out3, out4] = gpr\_simul(logtheta, covfunc, x, y, xstar, alpha, L)}


\textbf{Description}
\\ gpr\_simul - Gaussian process regression, with a named covariance function. Two
 modes are possible: training and prediction: if no test data are
given, the  function returns minus the log likelihood and its
partial derivatives with respect to the hyperparameters; this mode
is used to fit the hyperparameters. If test data are given, then
(marginal) Gaussian predictions are computed, whose mean and
variance are returned. Note that in cases where the covariance
function has noise contributions, the variance returned in S2 is
for noisy test targets; if you want the variance of the noise-free
latent function, you must substract the noise variance. The rutine
is modified to calculate the inverse covariance matrix only once
to speed-up simulation.
\\
\\ usage: [nlml dnlml] = gpr(logtheta, covfunc, x, y)
\\    or: [mu S2 alpha L]  = gpr(logtheta, covfunc, x, y, xstar, alpha, L)
\\
\\ where:
\\
\\   logtheta is a (column) vector of log hyperparameters
\\   covfunc  is the covariance function
\\   x        is a n by D matrix of training inputs
\\   y        is a (column) vector (of size n) of targets
\\   xstar    is a nn by D matrix of test inputs
\\   alpha    is inv(covariance matrix)y
\\   L        is cholesky(covariance matrix)
\\   nlml     is the returned value of the negative log marginal likelihood
\\   dnlml    is a (column) vector of partial derivatives of the negative
\\                 log marginal likelihood wrt each log hyperparameter
\\   mu       is a (column) vector (of size nn) of prediced means
\\   S2       is a (column) vector (of size nn) of predicted variances
\\
\\
\\ For more help on covariance functions, see "help covFunctions".
\\
\\ (C) copyright 2006 by Carl Edward Rasmussen (2006-03-20).
\\     modified  2009 by Jus Kocijan
