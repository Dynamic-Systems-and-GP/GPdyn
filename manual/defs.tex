%%%%%%%%%%%%%%%%%%%%%%%%%%%%%%%%%%%%%%%%%%%%%%%%%%%%%%%%%%%%%%%%%%%%%%%%%%%55

%% IZGLED
%% novej�i stil paragrafov
 \setlength{\parindent}{0mm}
 \setlength{\parskip}{11pt plus1pt minus1pt}
%% max �tevilo slik/stran
\setcounter{totalnumber}{2}
%% odstotek teksta
\renewcommand{\textfraction}{-1} %0.00

%% angle�ke besede
\newcommand{\ang}[1]{{angl. {\emph{#1}}}}



%%%%%%%%%%%%%%%%%%%%%%%%%%%%%%%%%%%%%%%%%%%%%%%%%%%%%%%%%%%%%%%%%%%%%%%%%%%55
%% DEFINICIJE OKOLIJ
\def\be{\begin{equation}}
\def\ee{\end{equation}}
\def\beq{\begin{eqnarray}}
\def\eeq{\end{eqnarray}}
\def\bit{\begin{itemize}}
\def\eit{\end{itemize}}
\def\ben{\begin{enumerate}}
\def\een{\end{enumerate}}
\def\bds{\begin{description}}
\def\eds{\end{description}}


\newcommand{\subsubsubsection}[1]{\vspace{8mm}{\emph{#1}}\\}






%%%%%%%%%%%%%%%%%%%%%%%%%%%%%%%%%%%%%%%%%%%%%%%%%%%%%%%%%%%%%%%%%%%%%%%%%%%55
%% DEFINICIJE ZA SLIKE
%%%%Uporaba:
%%\begin{figure}[hbt]
%%  \piceps{}
%%  \caption{}
%%  \label{}
%%\end{figure}
%%

% normal
\newcommand{\piceps}[1]{{
\centerline{\includegraphics[width=125mm]{#1}}}}  % 2/stran



%%%%%%%%%%%%%%%%%%%%%%%%%%%%%%%%%%%%%%%%%%%%%%%%%%%%%%%%%%%%%%%%%%%%%%%%%%%55
%% OKRAJSAVE v tekstu
\def\arrow{{ $\rightarrow$ }}
\def\stars{ \vspace{1cm} {\centerline{* * * * * * * * * * * * * * * * * * * *}} \vspace {1cm} }
\def\threestars{ \vspace{0.8cm} {\centerline{*  \hspace{1cm} * \hspace{1cm}   * }} \vspace {0.8cm} }

\def\tj{{tj.\,\,}}
%
\def\eg{{\emph{e.g.\,\,}}}


%\definecolor{name}{model}{specs}
\definecolor{gray}{rgb}{0.6,0.2,0.2}

%%%%%%
% moderna oblika to-do-ja
%\newcommand{\TODO}[1]{{\textbf{TODO: {#1}}}}
%\newcommand{\TODOK}[1]{\textcolor{blue}{{\textbf{TODO: \ {#1}}}}}

% komande za tab-e -- pisanje kode
\def\tab{{\hspace*{0.7cm}}}
\def\dtab{{\hspace*{1.4cm}}}
\def\ddtab{{\hspace*{2.1cm}}}



%%%%%%%%%%%%%%%%%%%%%%%%%% OPIS TOOLBOXA ///////////////////////////////////////
% izpis funkcije (uporaba v dodatku, kjer opisujem toolbox)
%\newcommand{\fun}[1]{{\emph{#1}}}
\newcommand{\fun}[1]{{\tt {#1}}}

% rule tabela razmik
\def\tabelarule{\rule[-3mm]{0mm}{10mm}}

% definicije gp
 \def\tboxlogtheta{{\tt logtheta} \dots vektor logaritma hiperparametrov $\log(\Th)$}
 \def\tboxcovfunc{{\tt covfunc} \dots uporabljena kovarian�na funkcija
 $C(\x_i,\x_j)$}
 \def\tboxinput{{\tt input} \dots $N \times D$ matrika u�nih vhodov $\XX$}
 \def\tboxtarget{{\tt target} \dots $N \times 1$ vektor u�nih izhodov $\y$}
 \def\tboxinputmat{{\tt inputmat} \dots $K \times D$ matrika vrednosti vhodov v GP
  model v posameznih korakih simulacije}
 \def\tboxlag{{\tt lag} \dots red avto-regresijskega (AR) dela modela}
 \def\tboxy{{\tt y} \dots napovedane srednje vrednosti izhodov modela v simulaciji}
 \def\tboxS2{{\tt S2} \dots napovedane variance izhoda}

%%%% definicije kovarian�ne funkcije
%% \def\tboxcovfx{{\tt x} \dots $n \times D$ matrika u�nih vhodov $\XX$}
%% \def\tboxcovfx{{\tt z} \dots $nn \times D$ matrika testnih izhodov, kjer vsaka vrstica predstavlja posamezen testni
%%  vhod $\x^*$}


 % definicije lmgp
 \def\tboxderiveinput{{\tt deriveinput} \dots $n_{eq} \times D$ matrika u�nih vhodov,
  ki dolo�ajo sredi��a lokalnih modelov}
 \def\tboxderivetarget{
  {\tt derivetarget} \dots $(N_{eq} \ D) \times 1$ vektor zdru�enih odvodov
  po posameznih komponentah {\tt derivetarget}=$[\th^{1T} \dots \th^{DT}]^T$}
  \def\tboxderivevar{
  {\tt derivevar} \dots $(n_eq D^2)  \times 1$ vektor sestavljen iz $ D^2  \times 1$
  vektorjev, od katerih vsak predstavlja navpi�no pobrano kovarian�no matriko parametrov posameznega lokalnega modela }








%
\newcommand{\Cmod}[1]{
{C_{\rm{mod}_{#1}}}}

%
\newcommand{\norm}{{\cal N}}
\newcommand{\lik}{{\cal L}}
\newcommand{\cov}{{\rm cov}}
\newcommand{\Emean}{{\mathbb{E}}}

%%\def\Ex{{\mathit{E}_\x}} %% Agatha
\def\Ex{\Emean}  %% Kristjan
\def\normcca{{\mathit{N}}}



% signali
\def\umin{{u_{\min}}}
\def\umax{{u_{\max}}}
\def\Umin{{U_{\min}}}
\def\Umax{{U_{\max}}}
\def\Tsw{{T_{sw}}}   % T_{\textrm{sw}}


% stanje histereze
\def\state{{\eta}}

% cenilke
\def\SE{{\mbox{SE}}}
\def\LD{{\mbox{LD}}}
\def\MRSE{{\mbox{MRSE}}}
\def\LL{{\mbox{LL}}}

%%%%%%%%%%%%%%%%%%%%%%%%%%%%%%%%%5
% FSGP model
% matrike
\def\fixA{{\tilde{\bf A}}}
\def\fixB{{\tilde{\bf B}}}
\def\fixC{{\tilde{\bf C}}}
\def\fixD{{\tilde{\bf D}}}
% odvajanja
\def\Nax{{\bi f_{\bi x0}}}
\def\Nau{{\bi f_{ u0}}}
\def\Nbx{{\bi g_{\bi x0}}}
%\def\Nbu{{\bi g_{\bi u0}}}
\def\Nbu{{\bi g_{u0}}}  %u ni� ve� vektor
\def\dx{{\delta \bi x}}
\def\du{{\delta u}} %% \def\du{{\delta \bi u}} % sedaj ni� ve� vektor!
% OLD FSGP
%\def \varth{{\rm \bf var}(\th)}
%\def \varthi{{\rm var}(\th^i)}
\def \varth{{\bs \sigma_\theta{^2}}(\bs \rho)}
\def \varthi{{\sigma^2_{\theta i}}(\bs \rho)}
% new just for m
\def \meanth{{\bs \mu_\theta}(\bs \rho)}
\def \meanthi{{\mu_{\theta i}}(\bs \rho)}
% short verzija brez nakazanosti, da parametri odvisni od \bs \rho
\def \meanthish{{\mu_{\theta i}}}
\def \varthish{{\sigma^2_{\theta i}}}
%%
\def \meanydot{{\mu_{\dot y}}}
\def \varydot{{\sigma^2_{\dot y}}}


% identifikacija �istilne naprave
\def\Sin{S_{\mbox{\footnotesize vh}}}
\def\Sout{S_{\mbox{\footnotesize izh}}}
\def\DO{\mbox{\emph{DO}}}

%%% FD fault detection
\def\vena{{\X{i}}}  % vektor enic dol�ine N
